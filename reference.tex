\documentclass[12pt,a4paper,]{article}
\usepackage[left=2cm, right=2cm, top=2cm]{geometry} 
\usepackage[utf8]{inputenc}
\usepackage[T1]{fontenc}  
\usepackage[italian]{babel}
\usepackage{amsmath}
\usepackage{amsfonts}
\usepackage{amssymb}
\usepackage{hyperref}
\usepackage{epstopdf}
\title{Fundamentals of Signals and Transmission Reference}
\date{}
\author{}
\begin{document}
\maketitle
\section{Convoluzione}
\subsection{Definizione}

\begin{align}
(f * g )(t) & \, \stackrel{\mathrm{def}}{=}\ \int_{-\infty}^\infty f(\tau) g(t - \tau) \, d\tau \\
& = \int_{-\infty}^\infty f(t-\tau) g(\tau)\, d\tau
\end{align}
\subsection{trucchi}
convolvere qualcosa con impulso => ritardare o anticipare della tau. dell'impulso\\
\begin{equation}
x(t) \ast A \delta(t-\tau) = Ax(t-\tau)
\end{equation}
convolvere impulso con se se setesso => raddoppiare freq. impulso \\
ATTENZIONE Convoluzione =! Moltiplicazione
\begin{equation}
x(t)\cdot\delta(t-\tau) = x(\tau)\cdot\delta(t-\tau)
\end{equation}

\section{Trasformata di Fourier}
\subsection{Definition}
\begin{equation}
H(f) = \int_{-\infty}^{\infty} \! h(\tau) e^{-j2\pi ft} \, \mathrm{d}\tau
\end{equation}
\begin{equation}
h(t) = \int_{-\infty}^{\infty} \! H(f) e^{j2\pi ft} \, \mathrm{d}f
\end{equation}
\subsection{Definizione discreta}
\begin{equation}
H(f) = \sum_{-\infty}^{\infty} \! h(nT) \,  e^{-j2\pi fnT} \, 
\end{equation}
\begin{equation}
h(nT) =T  \sum_{-\frac{1}{2T}}^{\frac{1}{2T}} \! H(f) \, e^{j2\pi fnT} \,
\end{equation}
\subsection{altre notazioni}
Esprimere la FT in modulo e fase (essendo una funzione complessa nella variabile f)
\begin{equation}
 F(f) = A(f) e^{i\varphi(f)}
\end{equation}
Dove:  $A(f) = \left| F(f)\right|$ è il modulo e $\varphi (f) = \arg \left( F(f) \right)$ è la fase. 


Then the inverse transform can be written:
\begin{equation}
f(t) = \int _{-\infty}^\infty A(f)\ e^{ j\left(2\pi f t +\phi (f)\right)}\,df
\end{equation}



\subsection{Properties}
 \begin{itemize}
 \item  \textbf{Dualità}: $ x(t) \longleftrightarrow X(f)$ $ X(f)\longleftrightarrow x(t)$
 \item \textbf{Scala}: $ x(\alpha t)  \longleftrightarrow \frac{1}{|\alpha|}X(\frac{f}{\alpha})$
 \item \textbf{Simmetria}
 \item Prodotto nei tempi è convoluzione nelle freq. e viceversa
 \item Moltiplicare per delta nelle frequenze 
 \item F(0) = tutta l'area  area sotto f(t)
  \end{itemize}
 \subsection{Trasformate notevoli}
 mettere il ritardo di tempo fatto bene|
 
 \begin{equation}
 \operatorname{sinc}(t) = \frac{\sin(\pi t)}{\pi t}  \longleftrightarrow  rect(f) 
  \end{equation}
   \begin{equation}
 \operatorname{sinc}(tA) =  \frac{\sin(\pi tA)}{\pi tA}  \longleftrightarrow  \frac{1}{|A|} rect\left(\frac{f}{A}\right)
 \end{equation}
  \begin{equation}
 \operatorname{sinc}^{2}(t)   \longleftrightarrow  tripulse(f)
  \end{equation}
 Treno di impulsi
 \begin{equation}
  \sum_{n=-\infty}^{\infty}  \delta(t-nT) \longleftrightarrow  \sum_{k=-\infty}^{\infty}  \frac{1}{T} \delta \left(f-\frac{k}{T}\right)
 \end{equation}
 
 \section{Cross-correlazione}
 \begin{align}
 R_{xy} & \, \stackrel{\mathrm{def}}{=}  \int_{-\infty}^{\infty} x(t+\tau)\ y^*(t) \,dt \\
 & = \int_{-\infty}^{\infty} X(f)\ Y^*(f)\ e^{j2\pi f \tau} \,df
 \end{align}
\section{Energy}
\begin{equation}
E= \int_{-\infty}^{\infty} \! |x(t)|^{2}\, \mathrm{d}t = \int_{-\infty}^{\infty} \! |X(f)|^{2}\, \mathrm{d}f
\end{equation}
\begin{equation}
E_{n}= \sum_{n= -\infty}^{\infty} \! |x(nT)|^{2}\, = T \int_{\frac{1}{T}} \! |X(f)|^{2}\, \mathrm{d}f mettere X tilde
\end{equation}
Leame tra energia segnale continuo e corrispettivo campionato:
\section{Power}
\begin{equation}
\lim_{T\to\infty} \frac{1}{T} \int_{-\frac{T}{2}}^{\frac{T}{2}} \! |x(t)|^{2}\, \mathrm{d}t 
\end{equation}
 \paragraph{Potenza segnale discreto}
 \begin{equation}
 \frac{Energia \: in \: un\:  periodo}{durata \: di \: un \: periodo}
 \end{equation}
 \section{Serie di Fourier}
 \begin{equation}
 \sum_{n=-\infty}^{\infty}  C_{n} e^{j2\pi\frac{n}{T_{0}}t}
  \end{equation}
  $C_{n}$ sono i coefficienti di Fourier
  
 \section{formule varie}
 \begin{equation}
 cos x = \operatorname{Re} \left(e^{jx}\right) =\frac{e^{jx} + e^{-jx}}{2}
  \end{equation}
   \begin{equation}
sin x = \operatorname{Im} \left(e^{jx}\right) =\frac{e^{jx} - e^{-jx}}{2j}
 \end{equation}
 
 
 
\end{document}
